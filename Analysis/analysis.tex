\chapter{Analysis}
\section{Introduction}
\subsection{Client Identification}
The system I plan to create is for Tylers Estate Agents. They are a local estate agents based in and around Cambridge and they have three offices, one in Cambridge and two in different villages. They have 20-30 employees who take clients on viewings around houses or work as receptionists. The company is run by three Directors who still take clients on viewings but also make company decisions. Every employee has their own desk with a PC to use e-mail and other computer applications. They are e-mailed or called to to tell them their appointments as they are booked, which they will write down in a diary that they have. The three offices normally get appointments for their local area, although they are all linked and employees often work in multiple offices. Tylers has recently celebrated it's twenty fifth anniversary, but it has been operating in a similar way for all of that time. There is a computer system used, although it is very basic and is mostly just for keeping diaries. Each of the offices is of a similar size and on the ground floor only. All the offices have televisions in the windows that show either a slide show of properties or a short video clip tour of a property.
\subsection{Define the current system}
The current system does not use a single computer program to organise appointments; instead the receptionist first takes the booking from the client in a phone call or a face-to-face conversation. This information includes: name, address, telephone number and desired timing. The receptionist writes it down in the computer diary, which can be seen throughout the Tylers computer network. These viewings are a free service. 

The receptionist e-mails, calls or asks an employee who is free whether they would take the appointment. The receptionist can see who's free from a timetable. However, appointments can be irregular and often are not written on the system. Once the employee is aware of the appointment, they write it in their diary either on paper or on their PC.

When the day of the appointment comes, the employee will look in there diary and see what appointments they have for the day. They then get to the appointment location early and wait for the client(s).


\subsection{Describe the problems}

\subsection{Section appendix}

\section{Investigation}

\subsection{The current system}

\subsubsection{Data sources and destinations}

\subsubsection{Algorithms}

\subsubsection{Data flow diagram}

\subsubsection{Input Forms, Output Forms, Report Formats}

\subsection{The proposed system}

\subsubsection{Data sources and destinations}

\subsubsection{Data flow diagram}

\subsubsection{Data dictionary}

\subsubsection{Volumetrics}

\section{Objectives}

\subsection{General Objectives}

\subsection{Specific Objectives}

\subsection{Core Objectives}

\subsection{Other Objectives}

\section{ER Diagrams and Descriptions}

\subsection{ER Diagram}

\subsection{Entity Descriptions}

\section{Object Analysis}

\subsection{Object Listing}

\subsection{Relationship diagrams}

\subsection{Class definitions}

\section{Other Abstractions and Graphs}

\section{Constraints}

\subsection{Hardware}

\subsection{Software}

\subsection{Time}

\subsection{User Knowledge}

\subsection{Access restrictions}

\section{Limitations}

\subsection{Areas which will not be included in computerisation}

\subsection{Areas considered for future computerisation}

\section{Solutions}

\subsection{Alternative solutions}

\subsection{Justification of chosen solution}
